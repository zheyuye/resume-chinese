%%%% 参考了https://www.wondercv.com/的模板

\documentclass[11pt]{article}

% disable indent globally
\setlength{\parindent}{0pt}
% some general improvements, defines the XeTeX logo
\usepackage{xltxtra}
% use hyperlink for email and url
\usepackage{hyperref}
\hypersetup{hidelinks}
\usepackage{url}
\urlstyle{tt}

\usepackage{xcolor}
%%%% 统一一种颜色,IC蓝
\definecolor{CVBlue}{RGB}{0,62,116}

%%% \widthof[]{} 用于特殊对齐时用到
\usepackage{calc}

%%%% 利用tikz来定位照片和学校Logo
\usepackage{graphicx}
\usepackage{tikz}
\usetikzlibrary{calc}

% loading fonts
\usepackage{fontspec}
\usepackage{xeCJK}
\CJKsetecglue{} %% 取消中文与数字之间间隙


%%%%% 字体需要自己下载安装,注意版权问题
%%%%% linux系统只需要字体路径就行了,如下
% % Main document font
\setmainfont[
    Path = Font/,
  Extension = .otf ,
  BoldFont = HelveticaNeueLTPro-Md.otf ,
]{HelveticaNeueLTPro-Roman.otf}

\setCJKmainfont[
Path = Font/,
  Extension = .ttf ,
BoldFont=Pro_GB18030DemiBold.ttf,
]{Pro_GB18030.ttf}

\usepackage{relsize}
\usepackage{xspace}
\protected\def\Cpp{{C\nolinebreak[4]\hspace{-.05em}\raisebox{.28ex}{\relsize{-1}++}}\xspace}

% use fontawesome
\usepackage{fontawesome}

\usepackage[
	a4paper,
	left=1.2cm,
	right=1.2cm,
	top=1.5cm,
	bottom=1cm,
	nohead
]{geometry}

\renewcommand{\baselinestretch}{1.2} %定义行间距1.2

\usepackage{titlesec}
\usepackage{enumitem}
\setlist{noitemsep} % removes spacing from items but leaves space around the whole list
%\setlist{nosep} % removes all vertical spacing within and around the list
\setlist[itemize]{topsep=0.25em, leftmargin=*}
\setlist[enumerate]{topsep=0.25em, leftmargin=*}



\titleformat{\section}         % Customise the \section command
  {\large\bfseries\raggedright} % Make the \section headers large (\Large),
                               % small capitals (\scshape) and left aligned (\raggedright)
  {}{0em}                      % Can be used to give a prefix to all sections, like 'Section ...'
  {}                           % Can be used to insert code before the heading
  [{\color{CVBlue}\titlerule}]                 % Inserts a horizontal line after the heading
\titlespacing*{\section}{0cm}{*1.6}{*1.2}

\begin{document}
\pagenumbering{gobble} % suppress displaying page number

%%%% 利用tikz来定位照片,部分招聘单位可能需要“以貌取人”
\begin{tikzpicture}[remember picture, overlay]
  \node[anchor = north east] at ($(current page.north east)+(-1.5cm,-1cm)$) {\includegraphics[height=3.5cm]{avatar}};
\end{tikzpicture}%
%%%% 利用tikz来定位学校Logo,这部分没有遵守College logo positioning
\begin{tikzpicture}[remember picture, overlay]
  \node[anchor = north west] at ($(current page.north west)+(0.45cm, -0.45cm)$) {\includegraphics[width=7cm]{imperial.png}};
\end{tikzpicture}%
%%%% 利用tikz来定位页脚栏,电子版简历使用,黑白纸质打印效果可能并不好。这里只在第一页显示,如果需要每页都有,页脚或者background中加入。
\begin{tikzpicture}[remember picture, overlay] 
  \node[anchor = south,fill=ICBlue,draw=none,minimum width=\paperwidth,minimum height=1.5em,align=center,font=\footnotesize,text=white] at ($(current page.south)$) {\faLinkedinSquare \ https://www.linkedin.com/in/username \qquad \faGithub \ https://github.com/username \qquad \faRssSquare \ http://blog.yours.me};
\end{tikzpicture}
%tikzpicture环境很敏感,注释周围的空格、空行都会引起水平距离或垂直距离的变化,
%

\vspace{3ex}
\centerline{\LARGE\bfseries{卷不过大家}}
\vspace{1ex}
\centerline{\normalsize{\faPhone\ 13xxx-xxx-xxx \quad \faEnvelopeO\ \href{zheyu.ye18@alumni.imperial.ac.uk}{zheyu.ye18@alumni.imperial.ac.uk}}}
\vspace{1.5ex}

% \centerline{\normalsize{}}
 
\section{\makebox[\widthof{\faGraduationCap}][c]{\color{CVBlue}\faGraduationCap}\  教育经历}

\textbf{哔哩哔哩在线教育} \hfill 2019.11 -- \makebox[\widthof{2099.11}][s]{至今}

刑法 \quad 罗翔说刑法 \hfill Online

\begin{itemize}
  \item 罗翔老师云学生
  \item 程序正义坚定拥护者
  \item 张三的共享辩护律师
\end{itemize}

\textbf{肯辛顿职业技术学院} \hfill 2018.09 -- 2019.11

MSc Advanced Computing \quad 硕士 \hfill {英国伦敦}
\begin{itemize}
  \item 勉强毕业
  \item 白读
\end{itemize}


\textbf{街道口男子技工学校} \hfill 2014.09 -- 2018.06

软件工程 \quad 本科 \quad 计算机科学与技术学院 \hfill 湖北武汉

\begin{itemize}
  \item GPA: 59/100
  \item 内卷竞争失败者
  \item 自信的真普通人
\end{itemize}


\section{\makebox[\widthof{\faGraduationCap}][c]{\color{CVBlue}\faSuitcase}\ 工作经历}

\textbf{血汗工厂}\  \hfill 20xx.xx -- 20xx.xx

建筑工程一线劳动者 \hfill 工地

努力在率先实现社会主义现代化上走在前列

\begin{itemize}
  \item 
  \begin{itemize}
      \item 
      \item 
      \item 
  \end{itemize}

\end{itemize}

\textbf{远洋邮轮1号}\  \hfill 20xx.xx -- 20xx.xx

实习 \quad 水手 \quad 花式摸鱼工程师  \hfill 精神家园

负责策划麦哲伦五百年纪念活动, 坚持不忘初心
\begin{itemize}
  \item 精神荷兰人, 海上马车夫复兴者
  \item 工人阶级的伟大代表 -- 不想当资本家的奴隶不是好工人
\end{itemize}

\section{\makebox[\widthof{\faGraduationCap}][c]{\color{CVBlue}\faWrench }\ 项目经历}

\textbf{火星文明探索计划}\  \hfill 20xx.xx -- 20xx.xx

混毕业的毕业设计 \quad 无人指导 \hfill The Pseudo Aeronautics and Space Administration

\begin{itemize}
  \item 
\end{itemize}

\textbf{人类文明史的沙盘重演}\  \hfill 20xx.xx -- 20xx.xx

Sid Meier's Civilization VI \quad Research Institution of Steam \hfill Online 

审阅人类文明历史发展, 探究其客观规律-- 从水下第一个生命的萌芽开始,到石器时代的巨型野兽,再到人类第一次直立行走.
\begin{itemize}
  \item Rise and Fall
  \item Gathering Storm
  \item Red Death
  \item New Frontier
\end{itemize}

\section{\makebox[\widthof{\faGraduationCap}][c]{\color{CVBlue}\faInstitution}\ 公开成果}

SSS: Deep Learning Based Social Scrap Selection Method

\textbf{Involutionist}

113rd Annual Conference on Neural Information Processing Systems (NeurIPS), 2099. (Accepted as Best Paper)

\href{}{[paper]} \href{}{[project page]} 

\section{\makebox[\widthof{\faGraduationCap}][c]{\color{CVBlue}\faCogs}\ 专业技能}

\begin{itemize}[parsep=0.5ex]
  \item 编程语言: Python $\gg$ C = CPP > Java
  \item 平台: 全栈工程师
  \item 阴阳怪气
\end{itemize}

\section{\makebox[\widthof{\faGraduationCap}][c]{\color{CVBlue}\faHeart}\ 获奖情况}
NeurIPS Best Paper Award \hfill 2099.12

\section{\makebox[\widthof{\faGraduationCap}][c]{\color{CVBlue}\faInfo}\ 其他}
% increase linespacing [parsep=0.5ex]
\begin{itemize}[parsep=0.5ex]
  \item 技术博客: http://blog.yours.me
  \item GitHub: https://github.com/username
  \item 语言: 英语 - 熟练(TOEFL xxx)
\end{itemize}

%%%% 如果多页简历,可以手动在适当位置插入 \newpage 或者 \clearpage 开始新一页
\begin{tikzpicture}[remember picture, overlay] 
  \node[anchor = south,fill=ICBlue,draw=none,minimum width=\paperwidth,minimum height=1.5em,align=center,font=\footnotesize,text=white] at ($(current page.south)$) {\faLinkedinSquare \ https://www.linkedin.com/in/username \qquad \faGithub \ https://github.com/username \qquad \faRssSquare \ http://blog.yours.me};
\end{tikzpicture}
\end{document}
